\documentclass[11pt,letterpaper]{article}
\usepackage{graphicx}
\usepackage[margin=1.25in]{geometry}
\usepackage{amsmath}
\usepackage{amsfonts}
\newcommand{\cov}{\operatorname{cov}}
\newcommand{\Var}{\operatorname{Var}}
\usepackage{float}

\begin{document}
\begin{center}
\textbf{Characteristic Function Evolution for Levy SDE}\\
Harish S. Bhat and Arnold D. Kim, 2018
\end{center}

\paragraph{Problem Statement.} Let $L_t^{\alpha}$ denote an $\alpha$-stable Levy processes, i.e., a process such that:
\begin{enumerate}
\item $L_0^\alpha = 0$ almost surely,
\item $L_t^\alpha$ has independent increments, and
\item For $t > s \geq 0$, $L_t^\alpha - L_s^\alpha \sim S_\alpha((t-s)^{1/\alpha},0,0)$.  That is, the increment over a time interval of length $t-s$ has an $\alpha$-stable distribution with scale parameter $\sigma = (t-s)^{1/\alpha}$, skewness parameter $\beta=0$, and location parameter $\mu=0$.  The characteristic function of this increment is:
\begin{equation}
\label{eqn:cfinc}
E[\exp(i s (L_t^\alpha - L_s^\alpha))] = \exp(-(t-s) |s|^\alpha).
\end{equation}
\end{enumerate}

\noindent Now consider the stochastic differential equation (SDE)
\begin{equation}
\label{eqn:sde}
dX_t = f(X_t) dt + g(X_t) dL_t^{\alpha}.
\end{equation}
Let $p(x,t)$ denote the probability density function (PDF) of $X_t$---note that $p$ is the exact PDF of the exact solution of the SDE.

\begin{quote}
\emph{Suppose that $p(x,0)$ is given.  Our goal is to compute $p(x,t)$ for $t > 0$.}
\end{quote}

\paragraph{Derivation of Method (Temporal Discretization).} To derive our method, we first discretize (\ref{eqn:sde}) in time via Euler-Maruyama with step $h > 0$:
\begin{equation}
\label{eqn:em}
x_{n+1} = x_n + f(x_{n}) h + g(x_{n}) \Delta L_{n+1}^{\alpha},
\end{equation}
where $\Delta L_{n+1}^{\alpha}$ is independent of $x_n$ and has characteristic function
\begin{equation}
\label{eqn:emchar}
\psi_{\Delta L^{\alpha}_{n+1}}(s) = \exp(- h |s|^\alpha).
\end{equation}
The drift $f$ and diffusion $g$ functions can be assumed to be smooth.  We can also assume that $g$ is bounded away from zero, i.e., that there exists $\delta > 0$ such that $|g(x)| \geq \delta$ for all $x$.  In fact, it is of interest to solve this problem (well) in the case where $g$ is a positive constant.

We let $\widetilde{p}(x,t_n)$ denote the exact PDF of $x_n$, itself an approximation to the exact solution at time $t_n$, $X(t_n)$.

Let us denote the conditional density of $x_{n+1}$ given $x_n = y$ by $p_{n+1, n}(x | y)$.  Applying this to (\ref{eqn:em}), we obtain the following evolution equation for the marginal density of $x_n$:
\begin{equation}
\label{dtq}
\widetilde{p}(x, t_{n+1})=\int_{-\infty}^\infty p_{n+1, n}(x | y ) \widetilde{p}(y, t_n) \, \, dy.
\end{equation}
\noindent
From (\ref{eqn:em}), we can show that the characteristic function of the conditional density $p_{n+1, n}(x | y )$  is
\[
e^{is \left(y+ f(y)h \right)}\exp (- h |s g(y)|^{\alpha} ).
\]
Therefore, we can compute the characteristic function using
\begin{equation}
\label{eqn:CFdefn}
\psi_{n+1}(s) = \int_{x=-\infty}^{\infty}e^{isx}\widetilde{p}(x,t_{n+1})dx.
\end{equation}
The characteristic function is given by
\begin{equation}
\label{eqn:CFupdate}
\psi_{n+1}(s) = \int_{y=-\infty}^{\infty}e^{is\left(y+f(y)h\right)}\exp{\left(  -h |s g(y)|^{\alpha} \right)}\widetilde{p}(y, t_n) \, dy.
\end{equation}
Since
\begin{equation}
\label{eqn:inverseFT}
\widetilde{p}(y, t_n) = \frac{1}{2\pi}\int_{u=-\infty}^{\infty}e^{-iuy}\psi_{n}(u)\, du
\end{equation}
from (\ref{eqn:CFupdate}) we get
\begin{equation}
\psi_{n+1}(s) =  \int_{u=-\infty}^{\infty}\left[\frac{1}{2\pi}\int_{y=-\infty}^{\infty}e^{is\left(y+f(y)h\right)}\exp{\left(  -h |s g(y)|^{\alpha} \right)}e^{-iuy}\, dy \right]\psi_{n}(u) \, du.\nonumber
\end{equation}
Let
\[
K(s,u) = \frac{1}{2\pi}\int_{y=-\infty}^{\infty}e^{is\left(y+f(y)h\right)}\exp{\left(  -h |s g(y)|^{\alpha} \right)}e^{-iuy}\, dy.
\]
and we get
\begin{equation}
\label{eqn:ctq}
\psi_{n+1}(s) =  \int_{u=-\infty}^{\infty}K(s,u)\psi_{n}(u) \, du.
\end{equation}
\emph{Up to quadrature, (\ref{eqn:ctq}) is the algorithm.} By repeatedly applying (\ref{eqn:ctq}) we evolve the characteristic function forward in time.  If, at any point, we want to retrieve the PDF from the characteristic function, we use (\ref{eqn:inverseFT}).

\paragraph{Numerical Analysis (Spatial Discretization).} The idea now is to spatially discretize both the characteristic function $\psi_0(s)$, the kernel $K(s,u)$, and the integral in (\ref{eqn:ctq}).   Numerical evaluation of the kernel is the most expensive and critical part, and so we focus on that first.

Before doing any numerical computation, we first note that the kernel is singular at $s=0$:
\begin{equation}
\label{eqn:s0}
K(0,u) = \frac{1}{2 \pi} \int_{y=-\infty}^\infty e^{-i u y} \, dy = \delta(u).
\end{equation}
If we now use this result in (\ref{eqn:ctq}), we obtain
\begin{equation}
\label{eqn:consprob}
\psi_{n+1}(0) = \int_{u=-\infty}^\infty \delta(u) \psi_n(u) \, dy = \psi_n(0).
\end{equation}
By (\ref{eqn:CFdefn}), we know that
\[
\psi_n(0) = \int_{x=-\infty}^\infty \widetilde{p}(x,t_n) \, dx.
\]
Hence (\ref{eqn:consprob}) is conservation of probability.  As long as our initial characteristic function satisfies $\psi_0(0) = 1$, all subsequent characteristic functions will preserve this property exactly.

In what follows, we assume $s \neq 0$.  We split the domain of integration:
\begin{multline*}
K(s,u) = \frac{1}{2 \pi} \int_{y=-\infty}^{-L/2} e^{is\left(y+f(y)h\right)}\exp{\left(  -h |s g(y)|^{\alpha} \right)}e^{-iuy}\, dy \\
 + \frac{1}{2 \pi} \int_{y=-L/2}^{L/2} e^{is\left(y+f(y)h\right)}\exp{\left(  -h |s g(y)|^{\alpha} \right)}e^{-iuy}\, dy \\
 + \frac{1}{2 \pi} \int_{y=L/2}^{\infty} e^{is\left(y+f(y)h\right)}\exp{\left(  -h |s g(y)|^{\alpha} \right)}e^{-iuy}\, dy
\end{multline*}
The inner integral over the finite domain $[-L/2, L/2]$ is the one we will compute using a quadrature rule.  We set up an equispaced grid with $N > 0$ grid points.  Then $\Delta y = L/N$ and $y_j = -L/2 + (\Delta y) j$ for $j = 0, 1, 2, \ldots, N-1$.  The inner integral then becomes a sum of terms of the form
\[
\int_{y_j}^{y_{j+1}} e^{i k y} \phi(y) \, dy.
\]
We now begin the derivation of the quadrature rule.  Assume $\phi$ is sufficiently smooth so that we can approximate it well using the Lagrange interpolant:
\[
\phi(y) \approx \frac{y - y_j}{\Delta y} \phi(y_j + \Delta y) - \frac{y - y_{j+1}}{\Delta y} \phi(y_j).
\]
Using this approximation, we derive the quadrature rule:
\begin{align}
\int_{y_j}^{y_{j+1}} e^{i k y} \phi(y) \, dy &\approx \int_{y_j}^{y_{j+1}} e^{i k y} \frac{y - y_j}{\Delta y} \phi(y_j + \Delta y) \, dy - \int_{y_j}^{y_{j+1}} e^{i k y} \frac{y - y_{j+1}}{\Delta y} \phi(y_j) \, dy \nonumber \\
\label{eqn:quadrule}
 &= \frac{m_1(y_j) - y_j m_0(y_j)}{\Delta y} \phi(y_j + \Delta y) - \frac{m_1(y_j) - y_{j+1} m_0(y_j)}{\Delta y} \phi(y_j) 
\end{align}
In this derivation, we have
\[
m_0(y_j) = \int_{y_j}^{y_{j+1}} e^{i k y} \, dy = \begin{cases} \Delta y & k=0 \\ (ik)^{-1} (e^{i k y_{j+1}} - e^{i k y_j}) & k \neq 0 \end{cases}
\]
and
\[
m_1(y_j) = \int_{y_j}^{y_{j+1}} y e^{i k y} \, dy = \begin{cases} \frac{1}{2} (y_{j+1}^2 - y_j^2) & k = 0 \\ (i k)^{-1}( y_{j+1} e^{i k y_{j+1}} - y_j e^{i k y_j}) + k^{-2} (e^{i k y_{j+1}} - e^{i k y_j}) & k \neq 0. \end{cases}
\]
Equipped with the quadrature rule, we set
\[
\phi(x) = \exp \left[ h (i s f(x) - |s g(x) |^\alpha ) \right]
\]
and complete the calculation:
\begin{align}
\frac{1}{2 \pi} \int_{y=-L/2}^{L/2} & e^{is\left(y+f(y)h\right)}\exp{\left(  -h |s g(y)|^{\alpha} \right)}e^{-iuy}\, dy \nonumber \\
&= \frac{1}{2 \pi} \sum_{j=0}^{N-1} \int_{y_j}^{y_{j+1}} e^{i (s-u) y} \phi(y) \, dy \nonumber \\
\label{eqn:quadcalc}
&\approx \frac{1}{2 \pi} \sum_{j=0}^{N-1} \frac{m_1(y_j) - y_j m_0(y_j)}{\Delta y} \phi(y_j + \Delta y) - \frac{m_1(y_j) - y_{j+1} m_0(y_j)}{\Delta y} \phi(y_j).
\end{align}

\paragraph{Asymptotics.} We now turn to the integrals over the unbounded domains $(-\infty, -L/2)$ and $(L/2, +\infty)$. Our strategy will be to use asymptotic approximations of $f$ and $g$ to compute the integrals. We will explain this strategy be example.

As a first problem, suppose that $\alpha = 1$ and
\begin{gather*}
f(x) = \tan^{-1} x \\
g(x) = \sqrt{1 + x^2}
\end{gather*}
We have
\[
f(x) \approx \pm \pi/2, \text{ as } x \to \pm \infty, \text{ respectively,}
\]
and
\[
g(x) \approx |x|, \text{ as } |x| \to \infty.
\]
We assume $L$ is sufficiently large and apply these approximations:
\begin{multline*}
\frac{1}{2 \pi} \int_{y=-\infty}^{-L/2} e^{is\left(y+f(y)h\right)}\exp{\left(  -h |s g(y)|^{\alpha} \right)}e^{-iuy}\, dy \approx \frac{1}{2 \pi} \int_{y=-\infty}^{-L/2} e^{i(s-u)y} e^{-i s h \pi/2} e^{h |s| y} \, dy \\
 = - \frac{i}{2 \pi} \frac{\exp \left( - (i/2)(h \pi s + L(s-u)) - (1/2)(h L |s|) \right)}{s - u - i h |s|}
\end{multline*}
and
\begin{multline*}
\int_{y=L/2}^{\infty} e^{is\left(y+f(y)h\right)}\exp{\left(  -h |s g(y)|^{\alpha} \right)}e^{-iuy}\, dy \approx \int_{y=L/2}^{\infty} e^{i(s-u)y} e^{i s h \pi/2} e^{-h |s| y} \, dy \\
 = \frac{i}{2 \pi} \frac{\exp \left( (i/2)(h \pi s + L(s-u)) - (1/2)(h L |s|) \right)}{s - u + i h |s|}.
\end{multline*}
The two asymptotic contributions are complex conjugates of one another---due to the symmetries of $f$ and $g$---and hence we obtain
\begin{equation}
\label{eqn:asympcalc}
\Re \left[ \frac{i}{\pi} \frac{\exp \left( (i/2)(h \pi s + L(s-u)) - (1/2)(h L |s|) \right)}{s - u + i h |s|} \right]
\end{equation}
as the total asymptotic contribution.

\paragraph{Brief Bit of Analysis.} Given any random variable $X$ with density $p(x)$, we can define the characteristic function as the Fourier transform:
\[
\psi(s) = \int_{x=-\infty}^\infty e^{i s x} p(x) \, dx.
\]
Hence
\[
\| \psi(s) \| = \left| \int_{x=-\infty}^\infty e^{i s x} p(x) \, dx. \right| \leq \int_{x=-\infty}^\infty p(x) \, dx = 1.
\]
because $|e^{i s x}| = 1$ and $p(x) \geq 0$.  Note that $\psi(0) = 1$, and so $\|\psi \|_\infty = 1$.

Now suppose in our algorithm that $\| \psi_n \|_\infty = 1$.  Then we have
\begin{align*}
\| \psi_{n+1} (s) \| &= \left| \int_u K(s,u) \psi_n(u) \, du \right| \\
 &\leq \int_u |K(s,u)| |\psi_n(u)| \, du \\
 &\leq \int_u |K(s,u)| \, du.
\end{align*}
Hence in order for $\| \psi_{n+1}(s) \| \leq 1$, it is sufficient that
\[
\leq \int_u |K(s,u)| \, du \leq 1.
\]
After computing $K(s,u)$ numerically, it is easy to check this condition.  Typically, we form a matrix $\kappa$ by setting $\kappa_{ij} = K(s_i, u_j)$.  One numerical approximation of the above sufficient condition is then
\[
\sum_{j} |\kappa_{ij}| (\Delta u) \leq 1
\]
or
\[
\| \kappa \|_\infty \leq (\Delta u)^{-1}.
\]

\paragraph{Physical Space and the Fractional Fokker-Planck.}


\end{document}






