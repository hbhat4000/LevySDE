\documentclass[11pt,letterpaper]{article}
\usepackage{graphicx}
\usepackage[margin=1.25in]{geometry}
\usepackage{amsmath}
\usepackage{amsfonts}
\newcommand{\cov}{\operatorname{cov}}
\newcommand{\Var}{\operatorname{Var}}
\usepackage{float}

\begin{document}
\begin{center}
\textbf{Characteristic Function Evolution for L\'{e}vy SDE}\\
Harish S. Bhat and Arnold D. Kim, 2018
\end{center}

\paragraph{Problem Statement.} Let $L_t^{\alpha}$ denote an $\alpha$-stable L\'{e}vy processes, i.e., a process such that:
\begin{enumerate}
\item $L_0^\alpha = 0$ almost surely,
\item $L_t^\alpha$ has independent increments, and
\item For $t > s \geq 0$, $L_t^\alpha - L_s^\alpha \sim S_\alpha((t-s)^{1/\alpha},0,0)$.  That is, the increment over a time interval of length $t-s$ has an $\alpha$-stable distribution with scale parameter $\sigma = (t-s)^{1/\alpha}$, skewness parameter $\beta=0$, and location parameter $\mu=0$.  The characteristic function of this increment is:
\begin{equation}
\label{eqn:cfinc}
E[\exp(i s (L_t^\alpha - L_s^\alpha))] = \exp(-(t-s) |s|^\alpha).
\end{equation}
\end{enumerate}

\noindent Now consider the stochastic differential equation (SDE)
\begin{equation}
\label{eqn:sde}
dX_t = f(X_t) dt + g(X_t) dL_t^{\alpha}.
\end{equation}
Let $p(x,t)$ denote the probability density function (PDF) of $X_t$---note that $p$ is the exact PDF of the exact solution of the SDE.

\begin{quote}
\emph{Suppose that $p(x,0)$ is given.  Our goal is to compute $p(x,t)$ for $t > 0$.}
\end{quote}

\paragraph{Brief Review of Characteristic Functions.} Given any random variable $X$ with density $p(x)$, we can define the characteristic function as the Fourier transform:
\[
\psi(s) = \int_{x=-\infty}^\infty e^{i s x} p(x) \, dx.
\]
Note that
\[
\psi(0) = \int_{x=-\infty}^\infty p(x) \, dx = 1.
\]
Using $|e^{i s x}| = 1$ and $p(x) \geq 0$, we have
\[
\| \psi(s) \| = \left| \int_{x=-\infty}^\infty e^{i s x} p(x) \, dx. \right| \leq \int_{x=-\infty}^\infty p(x) \, dx = 1,
\]
Because $\psi(0) = 1$, we see that $\|\psi \|_\infty = 1$.

\paragraph{Derivation of Method (Temporal Discretization).} To derive our method, we first discretize (\ref{eqn:sde}) in time via Euler-Maruyama with step $h > 0$:
\begin{equation}
\label{eqn:em}
x_{n+1} = x_n + f(x_{n}) h + g(x_{n}) \Delta L_{n+1}^{\alpha},
\end{equation}
where $\Delta L_{n+1}^{\alpha}$ is independent of $x_n$ and has characteristic function
\begin{equation}
\label{eqn:emchar}
\psi_{\Delta L^{\alpha}_{n+1}}(s) = \exp(- h |s|^\alpha).
\end{equation}
The drift $f$ and diffusion $g$ functions can be assumed to be smooth.  We can also assume that $g$ is bounded away from zero, i.e., that there exists $\delta > 0$ such that $|g(x)| \geq \delta$ for all $x$.  In fact, it is of interest to solve this problem (well) in the case where $g$ is a positive constant.

We let $\widetilde{p}(x,t_n)$ denote the exact PDF of $x_n$, itself an approximation to the exact solution at time $t_n$, $X(t_n)$.

Let us denote the conditional density of $x_{n+1}$ given $x_n = y$ by $p_{n+1, n}(x | y)$.  Applying this to (\ref{eqn:em}), we obtain the following evolution equation for the marginal density of $x_n$:
\begin{equation}
\label{dtq}
\widetilde{p}(x, t_{n+1})=\int_{-\infty}^\infty p_{n+1, n}(x | y ) \widetilde{p}(y, t_n) \, \, dy.
\end{equation}
\noindent
From (\ref{eqn:em}), we can show that the characteristic function of the conditional density $p_{n+1, n}(x | y )$  is
\[
e^{is \left(y+ f(y)h \right)}\exp (- h |s g(y)|^{\alpha} ).
\]
Therefore, we can compute the characteristic function using
\begin{equation}
\label{eqn:CFdefn}
\psi_{n+1}(s) = \int_{x=-\infty}^{\infty}e^{isx}\widetilde{p}(x,t_{n+1})dx.
\end{equation}
The characteristic function is given by
\begin{equation}
\label{eqn:CFupdate}
\psi_{n+1}(s) = \int_{y=-\infty}^{\infty}e^{is\left(y+f(y)h\right)}\exp{\left(  -h |s g(y)|^{\alpha} \right)}\widetilde{p}(y, t_n) \, dy.
\end{equation}
Since
\begin{equation}
\label{eqn:inverseFT}
\widetilde{p}(y, t_n) = \frac{1}{2\pi}\int_{u=-\infty}^{\infty}e^{-iuy}\psi_{n}(u)\, du
\end{equation}
from (\ref{eqn:CFupdate}) we get
\begin{equation}
\psi_{n+1}(s) =  \int_{u=-\infty}^{\infty}\left[\frac{1}{2\pi}\int_{y=-\infty}^{\infty}e^{is\left(y+f(y)h\right)}\exp{\left(  -h |s g(y)|^{\alpha} \right)}e^{-iuy}\, dy \right]\psi_{n}(u) \, du.\nonumber
\end{equation}
Defining
\[
\widetilde{K}(s,u) = \frac{1}{2\pi}\int_{y=-\infty}^{\infty}e^{is\left(y+f(y)h\right)}\exp{\left(  -h |s g(y)|^{\alpha} \right)}e^{-iuy}\, dy,
\]
we get
\begin{equation}
\label{eqn:ctq}
\psi_{n+1}(s) =  \int_{u=-\infty}^{\infty}\widetilde{K}(s,u)\psi_{n}(u) \, du.
\end{equation}
\emph{Up to quadrature, (\ref{eqn:ctq}) is the algorithm.} By repeatedly applying (\ref{eqn:ctq}) we evolve the characteristic function forward in time.  If, at any point, we want to retrieve the PDF from the characteristic function, we use (\ref{eqn:inverseFT}).

\paragraph{Collocation.} Any method to evaluate (\ref{eqn:ctq}) will require spatial discretization, i.e., a finite-dimensional approximation of $\psi_n$.  We have investigated thoroughly a trapezoidal discretization of the integral, using both equispaced and non-equispaced grids.  Such approaches suffer from the problem that for $|u| \neq 0$ sufficiently small, $|\psi_n(u)| > 1$, a fatal issue.

Let us consider a collocation method in which we approximate $\psi_n$ using a mixture of Gaussians:
\begin{equation}
\label{eqn:gaussmix}
\psi_n(u) \approx \sum_{m=-M}^{m=M} \gamma^n_m \exp \left( \frac{-(u - u_m)^2}{\zeta} \right)
\end{equation}
Here we take $u_m = m \Delta u$ for some $\Delta u > 0$.  There is a compatibility condition between $\Delta u$ and the parameter $\zeta > 0$---the Gaussians must overlap not too little and not too much.  (TODO: make this precise!)  Note that the normalization condition $\psi_n(0) = 1$ turns into
\begin{equation}
\label{eqn:collocnorm}
\sum_{m=-M}^{m=M} \gamma^n_m \exp \left( \frac{-u_m^2}{\zeta} \right) = 1
\end{equation}
We now use (\ref{eqn:gaussmix}) in (\ref{eqn:ctq}) to obtain
\begin{equation}
\label{eqn:ctqcolloc}
\sum_{\ell=-M}^{\ell=M} \gamma^{n+1}_{\ell} \exp \left( \frac{-(s - u_{\ell})^2}{\zeta} \right) = \int_{u=-\infty}^{\infty} \widetilde{K}(s,u) \sum_{m=-M}^{m=M} \gamma^n_m \exp \left( \frac{-(u - u_m)^2}{\zeta} \right) \, du.
\end{equation}
Using the definition of $\widetilde{K}$, we see that
\begin{multline}
\label{eqn:Kdef}
K(s, u_m) = \int_{u=-\infty}^{\infty} \widetilde{K}(s,u) \exp \left( \frac{-(u - u_m)^2}{\zeta} \right) \, du \\
= \frac{1}{2 \pi} \int_{y=-\infty}^\infty e^{i s (y + f(y) h) } \exp(-h |s g(y)|^\alpha) \int_{u=-\infty}^\infty e^{-i u y} \left( \frac{-(u - u_m)^2}{\zeta} \right) \, du \, dy \\
= \frac{\sqrt{\zeta}}{2 \sqrt{\pi}} \int_{y=-\infty}^\infty e^{i s (y + f(y) h) } \exp(-h |s g(y)|^\alpha -y^2 \zeta/4) e^{-i u_m y} \, dy
\end{multline}
We have carried out the integral over $u$ exactly.  Using this in (\ref{eqn:ctqcolloc}), we have
\begin{equation}
\label{eqn:ctqcolloc2}
\sum_{m'=-M}^{m'=M} \gamma^{n+1}_{m'} \exp \left( \frac{-(s - u_{m'})^2}{\zeta} \right) = \sum_{m=-M}^{m=M} K(s,u_m) \gamma^n_m
\end{equation}
We enforce this equality at the $2M+1$ points $s = u_{\ell}$ for $-M \leq \ell \leq M$.  Let
$$
C_{\ell, m} = \exp \left( \frac{-(u_{\ell} - u_{m})^2}{\zeta} \right).
$$
Similarly, let us write
$$
K_{\ell, m} = K(u_{\ell}, u_m).
$$
Then $C$ and $K$ are $(2M+1) \times (2M+1)$ matrices.  The update equation (\ref{eqn:ctqcolloc2}) becomes
\begin{equation}
\label{eqn:collocupdate}
\gamma^{n+1} = C^{-1} K \gamma^n.
\end{equation}


\paragraph{Numerical Analysis (Spatial Discretization).} The idea now is to compute $\widetilde{K}(s,u)$.  We split the domain of integration:
\begin{multline*}
K(s,u) = \frac{1}{2 \pi} \int_{y=-\infty}^{-L/2} e^{is\left(y+f(y)h\right)}\exp{\left(  -h |s g(y)|^{\alpha} -y^2 \zeta/4 \right)}e^{-iuy}\, dy \\
 + \frac{1}{2 \pi} \int_{y=-L/2}^{L/2} e^{is\left(y+f(y)h\right)}\exp{\left(  -h |s g(y)|^{\alpha} -y^2 \zeta/4 \right)}e^{-iuy}\, dy \\
 + \frac{1}{2 \pi} \int_{y=L/2}^{\infty} e^{is\left(y+f(y)h\right)}\exp{\left(  -h |s g(y)|^{\alpha} -y^2 \zeta/4 \right)}e^{-iuy}\, dy
\end{multline*}
The inner integral over the finite domain $[-L/2, L/2]$ is the one we will compute using a quadrature rule.  We set up an equispaced grid with $N > 0$ grid points.  Then $\Delta y = L/N$ and $y_j = -L/2 + (\Delta y) j$ for $j = 0, 1, 2, \ldots, N-1$.  The inner integral then becomes a sum of terms of the form
\[
\int_{y_j}^{y_{j+1}} e^{i k y} \phi(y) \, dy.
\]
We now begin the derivation of the quadrature rule.  Assume $\phi$ is sufficiently smooth so that we can approximate it well using the Lagrange interpolant:
\[
\phi(y) \approx \frac{y - y_j}{\Delta y} \phi(y_j + \Delta y) - \frac{y - y_{j+1}}{\Delta y} \phi(y_j).
\]
Using this approximation, we derive the quadrature rule:
\begin{align}
\int_{y_j}^{y_{j+1}} e^{i k y} \phi(y) \, dy &\approx \int_{y_j}^{y_{j+1}} e^{i k y} \frac{y - y_j}{\Delta y} \phi(y_j + \Delta y) \, dy - \int_{y_j}^{y_{j+1}} e^{i k y} \frac{y - y_{j+1}}{\Delta y} \phi(y_j) \, dy \nonumber \\
\label{eqn:quadrule}
 &= \frac{m_1(y_j) - y_j m_0(y_j)}{\Delta y} \phi(y_j + \Delta y) - \frac{m_1(y_j) - y_{j+1} m_0(y_j)}{\Delta y} \phi(y_j) 
\end{align}
In this derivation, we have
\[
m_0(y_j) = \int_{y_j}^{y_{j+1}} e^{i k y} \, dy = \begin{cases} \Delta y & k=0 \\ (ik)^{-1} (e^{i k y_{j+1}} - e^{i k y_j}) & k \neq 0 \end{cases}
\]
and
\[
m_1(y_j) = \int_{y_j}^{y_{j+1}} y e^{i k y} \, dy = \begin{cases} \frac{1}{2} (y_{j+1}^2 - y_j^2) & k = 0 \\ (i k)^{-1}( y_{j+1} e^{i k y_{j+1}} - y_j e^{i k y_j}) + k^{-2} (e^{i k y_{j+1}} - e^{i k y_j}) & k \neq 0. \end{cases}
\]
Equipped with the quadrature rule, we set
\[
\phi(x) = \exp \left[ h (i s f(x) - |s g(x) |^\alpha) -x^2 \zeta/4  \right]
\]
and complete the calculation:
\begin{align}
\frac{1}{2 \pi} \int_{y=-L/2}^{L/2} & e^{is\left(y+f(y)h\right)}\exp{\left(  -h |s g(y)|^{\alpha} -y^2 \zeta/4 \right)}e^{-iuy}\, dy \nonumber \\
&= \frac{1}{2 \pi} \sum_{j=0}^{N-1} \int_{y_j}^{y_{j+1}} e^{i (s-u) y} \phi(y) \, dy \nonumber \\
\label{eqn:quadcalc}
&\approx \frac{1}{2 \pi} \sum_{j=0}^{N-1} \frac{m_1(y_j) - y_j m_0(y_j)}{\Delta y} \phi(y_j + \Delta y) - \frac{m_1(y_j) - y_{j+1} m_0(y_j)}{\Delta y} \phi(y_j).
\end{align}

\paragraph{Asymptotics.} We now turn to the integrals over the unbounded domains $(-\infty, -L/2)$ and $(L/2, +\infty)$. Our strategy will be to use asymptotic approximations of $f$ and $g$ to compute the integrals. We will explain this strategy by example.

Suppose we have a potential $V(x)$---here we think of harmonic or anharmonic, single- or multiple-well potentials.  Then we can define
$$
f(x) = -\frac{dV}{dx}.
$$
Suppose that the potential is eventually linear, i.e., for $|x| \geq L/2$,
$$
V(x) = C |x| + D.
$$
This implies that for $|x| \geq L/2$,
$$
f(x) = - C \operatorname{sgn}(x).
$$
Further assume that the diffusion coefficient is constant, i.e.,
$$
g(x) = g > 0.
$$
Let $\operatorname{erfc}$ be the complementary error function defined by
\[
\operatorname{erfc}(z) = 1 - \frac{2}{\sqrt{\pi}} \int_0^z e^{-t^2} \, dt.
\]
Then the asymptotic contributions will be:
\begin{multline}
\frac{1}{2 \pi} \int_{y=-\infty}^{-L/2} e^{is\left(y+f(y)h\right)}\exp{\left(  -h |s g(y)|^{\alpha} -y^2 \zeta/4  \right)}e^{-iuy}\, dy \\
\approx \frac{1}{2 \pi} \int_{y=-\infty}^{-L/2} e^{i(s-u)y + i s C h} e^{-h |s|^\alpha g^\alpha -y^2 \zeta/4 } \, dy \\
 =  ( 2 \sqrt{\pi \zeta} )^{-1} 
 \exp \left( -h |s|^\alpha g^\alpha -\frac{(s-u)^2}{\zeta } + i s C h  \right) \operatorname{erfc}\left(\frac{\zeta  L-4 i (s-u)}{4 \sqrt{\zeta }}\right) 
\end{multline}
and
\begin{multline}
\frac{1}{2 \pi} \int_{y=L/2}^{\infty} e^{is\left(y+f(y)h\right)}\exp{\left(  -h |s g(y)|^{\alpha} -y^2 \zeta/4  \right)}e^{-iuy}\, dy \\
\approx \frac{1}{2 \pi} \int_{y=L/2}^{\infty} e^{i(s-u)y - i s C h} e^{-h |s|^\alpha g^\alpha -y^2 \zeta/4 } \, dy \\
 = ( 2 \sqrt{\pi \zeta} )^{-1} 
 \exp \left( -h |s|^\alpha g^\alpha -\frac{(s-u)^2}{\zeta } - i s C h  \right) \operatorname{erfc}\left(\frac{\zeta  L + 4 i (s-u)}{4 \sqrt{\zeta }}\right)
\end{multline}


\end{document}

   


